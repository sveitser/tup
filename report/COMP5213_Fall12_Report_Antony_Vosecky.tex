\documentclass[12pt,abstracton,a4paper]{scrartcl}
\usepackage{lmodern}
\usepackage{graphicx}
\usepackage{epstopdf}
\usepackage{amsmath}
\usepackage{amsfonts}
\usepackage{algorithm}
\usepackage{algorithmic}
\usepackage[colorlinks]{hyperref}
\usepackage[sorting=none]{biblatex}
\bibliography{bibliography.bib}

\title{Topic Models for Twitter User Profiling}
\author{
\begin{tabular}{cc}
    Jan Vosecky & Mathis Antony  \\
    \href{mailto:jvosecky@ust.hk}{jvosecky@ust.hk} &
    \href{mailto:mantony@ust.hk}{mantony@ust.hk}
\end{tabular}
\\ The Hong Kong University of Science and Technology 
\\ Clear Water Bay 
\\ Hong Kong} 
\begin{document}
\maketitle

\begin{abstract}
We discuss, evaluate and compared various topic models in regards to profiling microblog users according to their interests. A challenging aspect of the problem is the evolution of the users and the topics of interests in the general population. 
\end{abstract}

\section{Introduction}
Latent Dirichlet Allocation (LDA) is a topic model originally invented by Blei et al.\cite{Blei03} as a supervised batch learning model in which every observable word $w$ is supposed to have a hidden topic $k$ and is drawn from a corresponding hidden distribution over the vocabulary $\beta_k$. We use this as a starting point for our models. In general, the exact computation of the posterior in such probabilistic graphical models is intractable. Inference can however be done via approximate methods, such as  \textit{Gibbs-Sampling} \cite{Geman84} or variational inference \cite{Bishop06}.

Our goal is to construct a model for Twitter user profiling. The task at hand is similar to topic modeling but has a few special properties, which call for modifications of smoothed LDA or the author topic model:
\begin{itemize}
	\item Posts on twitter (referred to as ''tweets``) are very short as they are limited to 140 characters.
	\item Tweets have a single, observable author.
	\item Tweets are published one at a time.
\end{itemize}
%
Based on these observation we propose the following two distinct models for twitter user profiling shown in figure \ref{fig:plates}. The main difference between the two models is the Twitter Author Topic Model 1 (TATM1) assumes that each tweet is about a single topic (this is based on the observation of tweets consisting maximally of about two dozen words) whereas in the Twitter Author Topic Model 2 (TATM2) we have a topic assignment on a per word basis. 
%
\begin{figure}
	%\includegraphics[width=0.6\linewidth]{plates}
	\caption{Models in Plate Notation}
	\label{fig:plates}
\end{figure}
%
Both of these models can take advantage of same author correlations as the author of a tweet is always known. As our average twitter user is assumed to be alive and tweeting we are very interested in having an online inference method as opposed to a batch method or standard Gibbs Sampling. An online learning method for LDA, namely \textit{stochastic variational inference} has been developed recently \cite{Hoffman10,Hoffman12}. We use it as a basis to develop an online learning method for our models. The generative process assumed by our models is similar to the author topic model in \cite{Rosen04}. 

\section{Tweet Author-Topic Model (with per-word topics)}

\begin{figure}[tb]
    \centering
    \includegraphics[width=0.3\textwidth]{TATM2.eps}
    \caption{TATM2 in plate notation}
    \label{fig:TATM2}
\end{figure}

The model is shown in Figure \ref{fig:TATM2}. The generative process is as follows:

\begin{enumerate}
	\item For $k \in {1, \ldots, K}$:
	\begin{itemize}
		\item Draw topic $\beta_k \sim \text{Dir}_V(\eta)$.
	\end{itemize}
	\item For $a \in {1, \ldots, A}$:
	\begin{itemize}
		\item Draw author's topic proportions $\theta_a \sim \text{Dir}_K(\alpha)$.
	\end{itemize}
  \item For each document (tweet) $d \in {1, \ldots, D}$:
	\begin{itemize}
		\item For each word $w \in {1, \ldots, N}$:
		\begin{itemize}
			\item Draw topic assignment $z_{d,n} \sim \text{Mult}_K(\theta_a)$.
			\item Draw word $w_{d,n} \sim \text{Mult}_V(\beta_{z_{d,n}})$.
		\end{itemize}
	\end{itemize}
\end{enumerate}


Given the hyperparameters $\alpha$ and $\eta$, the joint distribution of topics $\beta$, author-topic mixture $\theta$, topic assignments $\mathbf{z}$ and words $\mathbf{w}$ is given by:

\begin{equation}
p(\theta,\beta,\mathbf{z},\mathbf{w}|\alpha,\eta) = p(\theta|\alpha) p(\beta|\eta) \prod_{n=1}^{N}{p(z_n|\theta)p(w_n|z_n,\beta_{z_{d,n}})}
\end{equation}


\subsection{Complete conditionals}

First, we denote $\theta_a$ to be the topic proportions of the author $a$ of document $d$, i.e. $\theta_a = \theta_{a_d}$.

\textbf{Local hidden variables.} The complete conditional of the topic assignment $z_{d,n}$ is a multinomial,

\begin{equation}
p(z_{d,n} = k | \theta_a, \beta, w_{d,n}) \propto \exp \{ \log \theta_{a,k} + \log \beta_{k,w_{d,n}} \}
\end{equation}

\textbf{Global hidden variables.} In contrast with standard LDA, the topic proportions are now a global hidden variable.
The complete conditional of the author-topic proportions is a posterior Dirichlet,

\begin{equation}
p(\theta_a | \beta, z_{d}) \propto \text{Dir} ( \alpha + \sum_{d \in D_a}{\sum_{n=1}^{N}{z^k_{d,n}}}),
\end{equation}

\noindent where $D_a$ is the set of documents whose author is $a$.

Finally, the complete conditional for the topic $\beta_k$ is also a posterior Dirichlet,

\begin{equation}
p(\beta_k | \mathbf{z}, \mathbf{w}) \propto \text{Dir} ( \eta + \sum_{d=1}^{D}{\sum_{n=1}^{N}{z^k_{d,n}w_{d,n}}})
\end{equation}



\subsection{Variational parameters for batch variational inference}

The variational parameters are:
\begin{itemize}
	\item Global per-topic Dirichlets $\lambda_{1:K}$
	\item Global per-author Dirichlets $\gamma_{1:A}$
	\item Local per-word multinomials $\phi_{1:D,1:N}$
\end{itemize}

Each update of the local variables is defined as
\begin{equation}
\phi_{d,n} \propto \exp \{ \Psi(\gamma_a) + \Psi(\lambda_{.,w_{d,n}}) - \Psi(\sum_{v}{\lambda_{.,w_{d,n}}}) \}
\end{equation}

Each update of the global variational Dirichlets is defined as
\begin{equation}
\gamma_a = \alpha + \sum_{d \in D_a}{\sum_{n=1}^{N}{\phi_{d,n}}}
\end{equation}

\begin{equation}
\lambda_k = \eta + \sum_{d=1}^{D_a}{\sum_{n=1}^{N}{\phi^k_{d,n}w_{d,n}}}
\end{equation}



\subsection{Stochastic variational inference}

Each update of the local variables is defined as
\begin{equation}
\phi^k_{d,n} \propto \exp \{ \mathbb{E}[\log \theta_{a,k}] + \mathbb{E}[\log \beta_{k,w_{d,n}}] \}
\end{equation}

During the updates of local variables, we utilize an intermediate local variable $\hat{\gamma}_d$, which serves as an indicator of convergence of $\phi_{d}$ during the inner loop of the E-step. Specifically,
\begin{equation}
\hat{\gamma}_d = \sum_{n=1}^{N}{\phi_{d,n}}
\end{equation}


After fitting the local variables, we set the intermediate topics as
\begin{equation}
\hat{\lambda}_k = \eta + D \sum_{d=1}^{D_a}{\sum_{n=1}^{N}{\phi^k_{d,n}w_{d,n}}}
\end{equation}

Finally, the global topics and the global per-author distributions are updated as
\begin{equation}
\lambda^{t+1}_k = (1 - \rho_t) \lambda^{t} + \rho_t \hat{\lambda}_k
\end{equation}

\begin{equation}
\gamma^{t+1}_a = \gamma^{t}_a + \hat{\gamma}_d
\end{equation}

\printbibliography
\end{document}
