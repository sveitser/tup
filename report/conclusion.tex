\section{Conclusion}
We have set out with the goal of creating a model to determine the interests of twitter users based on their tweets. Due to Twitter specific properties and for reasons of scalability we require an online learning algorithm. Based on magnificent work by Blei, Hoffman and others in the field of topic modelling we were able to create two similar models that solved the task at hand reasonably well.

Finally we note that this was arguably too challenging and laborious a project to complete in one semester without much prior knowledge about the topic. As a result there is lot left to be done. Ideally we would use a hierarchical Dirichlet Process model allowing for discovery of the number of topics and creation of new topics when necessary.

We're also not fully utilizing the available information from Twitter as we have completely ignored the followers graph. It is very reasonable to assume a Twitter user follows people who tweet on topics he or she is interested about. One should therefore be able to leverage correlations between the interests of ``followees'' and their followers.

\paragraph{Contribution.}
Both authors contributed to literature research, the code base and report writing in roughly equal parts.